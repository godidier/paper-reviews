\documentclass{article}

\begin{document}
\title{Paper Reviews}
%\author{Didier Gohourou\\ Semlab Ritsumeikan University}
\date{2020-08-07}
\maketitle

\section*{Review 1}
Liao Lizi, Ryuichi Takanobu, Yunshan Ma, Xun Yang, Minlie Huang, 
and Tat-Seng Chua. 2019. "Deep Conversational Recommender in Travel," 
June. http://arxiv.org/abs/1907.00710.

\subsection*{Summary}
The need for conversational agents have proved to be a necessity in the 
travelling business especially for venue recommendation. Building appropriate 
conversational agent models is still a challenge. On one hand existing models 
are generaly neural conversational models that use an end to end approaches. 
Such models are good at capturing the local structure of word sequence 
but faces difficulty in remembering the global semantic structure of a 
dialog session. On the other hand, methods for venue recommendation show
limitation from their reliance on the exact match of constraints which is 
sensitive to language variation, weak relationship models between venues 
and the inability to backpropagate error signals from the end output to 
the raw input.

This study proposes a Deep Conversational Recommender System, to address the 
above mentionned limitation. It does it by using a combination of a 
topic based model and a Graph Convolutional Network (GCN) based model. 
The topic based component allows the agent to differentiate sub-topic 
while extracting a global semantic. The GCN helps the conversational 
recommender with better representation of venues.

The proposed DCR model have been compared to related state-of-the-art 
models including HRED that is based on Reccurent Neural Network to 
predict utterances based on previous one in the discussion, MultiWOZ that 
use a seq2seq model to map the dialog as a context to response problem, 
TopicRNN that use a seq2seq model that incorporate topics information, 
ReDial that is based on HRED and have a recommendation mechanism that 
considers the co-occurence of items, NCF that uses deep learning in 
a collaborative filtering mechanism. 

Automatted evaluation mechanisms like BLEU score and Entity Accuracy 
were used to rate the models. A Human evaluation based on criteria such 
as Fluency, Informativeness and Ranking were also used to evaluate the models. 
DCR proved to outperform all related models. 

Future work include boosting the performance of response generation, and 
leverage venue adoption data from travel e-commerce sites.

\subsection*{Comments}
This work is in many aspects similar to our initial research proposal, and 
more over to our current topic. Reproducing this study will be a good 
starting point for our own research topic titled Graph Learning for 
Conversational Recommender Systems.  An interesting direction in which 
we can extends this work is applying the findings to some other 
domain of interest (business, marketing, shopping, etc), and then 
find a way to generalize the model.

\section*{Review 2}
Lin Y.-L., Tran S., Davis L.S., 2019. Fashion Outfit Complementary 
Item Retrieval.

\subsection*{Summary}
Outfit complementary item retrieval consists of finding compatible item(s) 
to complete a fashion outfit. It can drastically improve the experience of 
online fashion customer and create more revenue for the retailers.
Previous work address this task by either with pairwise complementary 
retrieval, different from an entire outfit compatibility; or with an 
outfit compatibility prediction mechanism, but not in a retrieval setting.
This work presents a model for complementary fashion recommendation in a 
retrieval setting. The model is a scalable category-based attention selection, 
with a novel outfit ranking loss function.

The proposed model has three inputs: a source image, its category and a 
target category. A Convolutional Neural Network extracts features from the 
image. Multiple subspace embedding are generated by applying a set of masks 
to the image feature vector. The two categorical vectors are concactenated and 
fed into a sub network to predict the attention weights, for the final 
embedding computation.

The proposed framework is evaluated on Outfit complementary item retrieval
 and fill in the blank (FITB) tasks against related state-of-the art 
models that include: Siamese-Net, Type-aware and SCE-Net.
The proposed model show superior score compared to all other models used in 
the study, on both tasks: FITB and outfit compatibility.


\subsection*{Comments}
The study addresses a relevant commercial problem for online fashion retailer 
and propose a specific solution which is a better approach compared to 
existing solutions. 

In the paper, the model is theoretically well detailled. Practical
implementation details such as the deep learning framework use and other
implementation and runtime environment description would have been welcome. 

Despite the specifity of the model to the problem, the study in our
interests because we can get insights on building an item set recommendation 
mechanism in a retrieval setting.

%\section*{Review 3}
%(Citation here) 

%\subsection*{Summary}
%(Summary here)

%\subsection*{Comments}
%(Comments here)

\end{document}
