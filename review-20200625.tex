\documentclass{article}

\begin{document}
\title{Paper Reviews}
%\author{Didier Gohourou}
\date{yyyy-mm-dd}
\maketitle

\section*{Review}
. L. Herlocker, J. A. Konstan, and J. Riedl. Explaining collaborative
filtering recommendations. In CSCW’00, pages 241–250, 2000.

\subsection*{Summary}
In contrast to content based filtering, Automated collaborative filtering
(ACF) systems recommends items to a user based on the affinity to a similar
community of users to that item. ACF comes with many advantages such as
the ability to recommend any type of items since despite how difficult it
can be to find the appropriate features for a particular item in a content
based filtering system, this problem is inexistant for ACF. However,
ACF comes with some disadvantages such as the ones caused by cold-start
and the filter bubble phenomenon. For those reasons ACF were mostly used
to recommend low risk items. 

This paper explore the benefit of building an explanation mechanism within
automated collaborative filtering systems. This study is done by answering
three questions including: 
\begin{itemize}
    \item What model and techniques are effective in supporting explanation in an ACF system ?
    \item Can explanation facilities increase the acceptance of automated collaborative systems ? 
    \item Can explanation facilities increase the filtering performance of ACF system users ? 
\end{itemize}

Two explanation mechanism were described: The white box conceptual model
where most of the recommendation process is explained to the user, and the
black box model where only a confidence score is given to the user for a
particular recommendation.

Two  experiments were conducted and showed that providing explanations
increase the acceptance of the automated collaborative systems, and tend
to increase the filtering performance by making the user more confident
on they recommended choices.

\subsection*{Comments}

While showing the benefits of an explanation mechanism for the an automated
collaborative filtering, The paper describe how to implement such mechanism
and what to consider when doing so. 

Another issue, partly notified in the paper, emerge from providing
explanations: how to properly present them so that most users have an easy
understanding. Future work plan to address it as well as studying the reaction
of the users to informational explanations versus persuasive explanations.

\end{document}
