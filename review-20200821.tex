\documentclass{article}

\begin{document}
\title{Paper Reviews}
%\author{Didier Gohourou}
\date{2020-08-21}
\maketitle

\section*{Review 1}
Perozzi, B., Al-Rfou, R., \& Skiena, S. (2014). DeepWalk: Online learning 
of social representations. In Proceedings of the ACM SIGKDD International 
Conference on Knowledge Discovery and Data Mining (pp. 701–710). 
Association for Computing Machinery. https://doi.org/10.1145/2623330.2623732

\subsection*{Summary}
The sparsity of a network representation makes it difficult apply statistical
learning algorithms. This paper introduces DeepWalk. A novel approach 
to use models from the Natural Language Processing to learn on graph 
structured data. 

The model works by generating sequences of nodes using random walks and 
use Skipgram, a model originating from NLP, to learn latent vector 
representation of nodes. Those representations can later be used for 
downstream tasks, the main being node classification task for social 
network. Deepwalk shows advantages such as efficient computation by using 
negative sampling, being highly paralellizable, and the ability to be 
trained online. 

Experiments were conducted to evaluate DeepWalk against related models 
including \textit{SpectralClustering, Modularity, EdgeCluster, wvRN, Majority}.
The experiments were conducted using datasets from BlogCatalog a network 
of social relationships among bloggers, Flickr a network of contact among 
photo sharing users, and Youtube a social network of users of vidoe sharing. 
DeepWalk outperformed the other models on the social network classification 
taks when performance are measured using an F1 score. 


\subsection*{Comments}
The paper introduced a novel approach, and is among the first of using 
 Deep Learning methods, especially the one from  NLP on graph structured 
data for learning. Despite being published in 2014, the paper became 
popular starting 2017 when the deep learning community showed interest in 
learing from graph structured data. This work is still used as benchmark 
for recent paper and reviews in the emerging field of graph learning. 

An intuitive comprehension of this work will help us understand how recent 
work build upon it and in which direction progress were made in graph 
learning and also in which direction progress is still to be made.



\section*{Review 2}
Kipf, T. N., \& Welling, M. (2017). Semi-supervised classification with 
graph convolutional networks. In 5th International Conference on Learning 
Representations, ICLR 2017 - Conference Track Proceedings. International 
Conference on Learning Representations, ICLR.


\subsection*{Summary}
In the emerging field of graph learning a common problem is node 
classification where a small subset of nodes are labeled (classified). 
This paper propose a model that uses convolution operations in the solution.
The study contribute by firstly introducing a layer-wise propagation rule 
for neural network models and operate directly on graphs, and by secondly 
demonstrating how this form of graph based neural network model can be 
used for fast and scalable semi-supervised classification of nodes in graph.

The theoretical motivation of the proposed model are fast approximate 
convolutions that include the use of mechanisms such as spectral graph 
convolution which can be defined as a multiplication of a signal with a 
parametrized filter, and a layer-wise linear model that consist of stacking
multiple convolutional layer.

The proposed model named Graph Convolutional Network (GCN) was evaluated 
against the following: \textit{ManiReg, SemiEmb, LP, DeepWalk, ICA, Planetoid}. 
The experiment was ran using datasets from citation networks: Citeseer, 
Cora, Pubmed and a knowledge graph (NELL). GCN outperform all benchmark 
model on the percentage of accuracy in a node classificaiton task.

Future work aim at addressing the current limitation of the model that are 
a memory requirement that grows linearly with the size of the dataset, 
the lack of support for directed edges and edge features, and the limiting
assuptions.


\subsection*{Comments}
Graph Convolutional Network is currently among the go-to model for graph 
learning related task and is still to some extend considered state of the 
art for semi-supervised node classification. 

Learning from a graph structured data using a convolution mechanism might 
seem natural at first, but the authors showed that with prior mathematical
preprocessing of some representation and properties of a graph such as
the graph laplacian (computed from the adjacency and the degree matrix)
we can obtain results better than previous state of the art.
However due to the extensive mathematical preprocessing, the model
might not feel very intuitive.

\end{document}
