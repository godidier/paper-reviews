\documentclass{article}

\begin{document}
\title{Paper Reviews}
%\author{Didier Gohourou}
\date{2020-05-08}
\maketitle

\section*{Review 1}
L. Chen and P. Pu. Critiquing-based recommenders: survey and emerging trends.
 User Modeling and User-Adapted Interaction, 22(1-2):125–150, 2012.

\subsection*{Summary}

Among different  recommender system types, the ones that elicit users’ feedback
made on the recommended item are called criquiting-based. Critiquing-based 
recommenders help users incrementally build their preference models and refine
them as they see more options. This paper is a survey of critiquing-based
recommenders systems type. It explains that such systems emerged to overcome
cold start problems faced by conversational recommender systems. Then it
identifies three types of critiquing systems, that include natural language
dialog based systems that prompt the users to provide preferences feedback
using natural language, system-suggested critiquing based systems that
proactively generates knowledge-based critiques for the users to select, and
user-initiated critiquing systems that stimulate the users to make
self-motivated critiques. A hybrid approach is also presented (combination
 of different critiquing-based style). 

\subsection*{Comments}

The paper shows that critiquing-based recommender systems are well received
in the industry and power notable recommendation platforms. Future trends of
the field are also discussed. They include the  hybrid systems previously
discussed, adaptive element that consist of adapting the critiquing process
to the users’ changing preferences, and critiquing system for low-involvement
product domains  that consist of building critiquing based recommender systems
for low risk/value goods and letting the system more aggressive in generating
system-based critiques.

We noticed that the authors disseminated their work within the paper, while
doing the review. This shows how relevant is their work within the field
compared to the state of the art. This is a well though practice that will
help people researching the field noticing your contribution on the field
as overall.


\section*{Review 2}
D. Contreras, M. Salamó, I. Rodríguez, and A. Puig. A 3D visual interface for
critiquing-based recommenders: Architecture and interaction.
IJIMAI, 3:7–15, 2015

\subsection*{Summary}

This paper proposes a Collaborative Conversational Recommender system that
users experience in a 3D environment. The system consist of a three layers
including a 3D Collaborative Space Client where users can interact with each
others and with the recommender system, a Collaborative Conversational
Recommender layer that contain the recommendation process (algorithms, users
model, etc.), and a 3D Collaborative Space Server that is the communication
layer between the previous two.

The proposed system was evaluated by 20 participants who noted a positive
experience, by answering a 20 questions questionnaire, after using the system.
Among them 70\% mentioned that the system accurately recommended them what
they were looking for. 

\subsection*{Comments}

Having the user explore a 3D environment is a novel idea that should be
explore with care. The common user of ecommerce platforms access them via
a web browser or a smartphone application, and navigating a 3D environmnent
to find products through those media can be cumbersome. Yet this proposal
can be a great opportunity for the video game market and new devices like
Augmented and  Virtual Reality headsets that offers enormous possibilities
to the proposed system.

\end{document}
